% TeX encoding = utf8
% TeX spellcheck = pl_PL 
\documentclass[a4paper,titlepage,11pt,twosides,floatssmall]{mwrep}
\usepackage[left=2.5cm,right=2.5cm,top=2.5cm,bottom=2.5cm]{geometry}
\usepackage[OT1]{fontenc}
\usepackage{polski}
\usepackage{amsmath}
\usepackage{amsfonts}
\usepackage{amssymb}
\usepackage{graphicx}
\usepackage{url}
\usepackage{tikz}
\usetikzlibrary{arrows,calc,decorations.markings,math,arrows.meta}
\usepackage{rotating}
\usepackage[percent]{overpic}
\usepackage[utf8]{inputenc}
\usepackage{xcolor}
\usepackage{pgfplots}
\usetikzlibrary{pgfplots.groupplots}
\usepackage{listings}
\usepackage{matlab-prettifier}
\usepackage{siunitx}
\usepackage[section]{placeins}
\definecolor{szary}{rgb}{0.95,0.95,0.95}
\SendSettingsToPgf
\sisetup{detect-weight,exponent-product=\cdot,output-decimal-marker={,},per-mode=symbol,binary-units=true,range-phrase={-},range-units=single}

%konfiguracje pakietu listings
\lstset{
	backgroundcolor=\color{szary},
	frame=single,
	breaklines=true,
}
\lstdefinestyle{customlatex}{
	basicstyle=\footnotesize\ttfamily,
	%basicstyle=\small\ttfamily,
}
\lstdefinestyle{customc}{
	breaklines=true,
	frame=tb,
	language=C,
	xleftmargin=0pt,
	showstringspaces=false,
	basicstyle=\small\ttfamily,
	keywordstyle=\bfseries\color{green!40!black},
	commentstyle=\itshape\color{purple!40!black},
	identifierstyle=\color{blue},
	stringstyle=\color{orange},
}
\lstdefinestyle{custommatlab}{
	captionpos=t,
	breaklines=true,
	frame=tb,
	xleftmargin=0pt,
	language=matlab,
	showstringspaces=false,
	%basicstyle=\footnotesize\ttfamily,
	basicstyle=\scriptsize\ttfamily,
	keywordstyle=\bfseries\color{green!40!black},
	commentstyle=\itshape\color{purple!40!black},
	identifierstyle=\color{blue},
	stringstyle=\color{orange},
}

%wymiar tekstu (bez żywej paginy)
\textwidth 160mm \textheight 247mm

%ustawienia pakietu pgfplots
\pgfplotsset{
	tick label style={font=\scriptsize},
	label style={font=\small},
	legend style={font=\small},
	title style={font=\small}
}

\def\figurename{Rys.}
\def\tablename{Tab.}

%konfiguracja liczby pływających elementów
\setcounter{topnumber}{0}%2
\setcounter{bottomnumber}{3}%1
\setcounter{totalnumber}{5}%3
\renewcommand{\textfraction}{0.01}%0.2
\renewcommand{\topfraction}{0.95}%0.7
\renewcommand{\bottomfraction}{0.95}%0.3
\renewcommand{\floatpagefraction}{0.35}%0.5

\begin{document}
	
	\begin{titlepage}
		\begin{center}
			\Huge{\textsc{Sprawozdanie z drugiego projektu z przedmiotu \\,,Sztuczna Inteligencja w Automatyce''}} \\
			[15cm]
			\Large{Numer zadania: 10 \\Wykonawcy:}\\
			\Large{Daniel Giełdowski \\ Piort Chachuła}
		\end{center}
	\end{titlepage}
	
	\tableofcontents
	\newpage
	\chapter{Zadanie 1}
	\label{ch:z1}
	\begin{equation}
	\left\{
	\begin{tabular}{l}
		$x_1(k)= - \alpha_1 x_1(k-1) + x_2(k-1) + \beta_1 g_1(u(k-3))$ \\\\
		$x_2(k)= - \alpha_2 x_1(k-1) + \beta_2 g_1(u(k-3))$\\\\
		$y(k)=g_2(x_1(k))$\\\\
	\end{tabular}
	\right.
	\label{eq:dyskretne}
	\end{equation}
	gdzie $u$-sygnał wejściowy, $y$-sygnał wyjściowy, $x_1$, $x_2$ - zmienne stanu, $\alpha_1 = -1,422574$, $\alpha_2 = 0,466776$, $\beta_1 = 0,017421$, $\beta_2 = 0,013521$ oraz
	\begin{equation}
		g_1(u(k-3))=\frac{exp(5u(k-3))-1}{exp(5u(k-3))+1}, g_2(x_1(k))=1-exp(-1.5x_1(k))
	\end{equation}
	Punkt pracy $u=y=x_1=x_2=0$, $u^min=-1 : u^max = 1$.
	W wersji statycznej:
	\begin{equation}
		\left\{
		\begin{tabular}{l}
		$x_1= - \alpha_1 x_1+x_2+ \beta_1 g_1(u)$ \\\\
		$x_2= - \alpha_2 x_1+ \beta_2 g_1(u)$\\\\
		$y=g_2(x_1)$\\\\
		\end{tabular}
		\right.
		\label{eq:statyczne}
	\end{equation}
	Po przekształceniach:
	\begin{equation}
		x_1 = \frac{(\beta_1 + \beta_2)g_1(u)}{1+\alpha_1+\alpha_2}
		\label{eq:x1_static}
	\end{equation}
	Podstawieniu równania (\ref{eq:x1_static}) do $y$ otrzymujemy
	\begin{equation}
		y(u) = g_2(\frac{(\beta_1 + \beta_2)g_1(u)}{1+\alpha_1+\alpha_2}
		\label{eq:x1_static})
	\end{equation}
	\chapter{Modelowanie procesu}
	\label{ch:mod}
	
	\section{Opóźnienie}
		\label{sec:tau}
	
	\section{Dobór liczby neuronów}
		\label{sec:neurony}
		
	\section{Model z algorytmu BFGS}
		\label{sec:bfgs}
		
	\section{Symulacja modelu z algorytmu BFGS}
		\label{sec:bfgs_sym}
		
	\section{Model z algorytmu najszybszego spadku}
		\label{sec:naj_sp}
		
	\section{Model z algorytmu BFGS z uczeniem bez rekurencji}
		\label{sec:bfgs_arx}
		
	\section{Symulacja modelu z algorytmu BFGS z uczeniem bez rekurencji}
		\label{sec:bfgs_arx_sym}
		
	\section{Model metodą najmniejszych kwadratów}
		\label{sec:mnk}
	
\end{document}