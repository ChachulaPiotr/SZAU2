\chapter{Używane skrypty}
	\textbf{Do wykonania powyższych zadań wykorzystane zostały następujące skrypty:}
	\begin{itemize}
		\item Charakterystyka statyczna - $charakterystyka\_statyczna.m$
		\item Dane uczące i weryfikujące - $generowanie\_danych.m$ oraz $wykres\_danych.m$
		\item Dobór liczby neuronów - $modelowanie.m$ wraz z generowanym plikiem $osiagi.txt$
		\item Trenowanie różnych modeli neuronowych - $naucz\_model.m$
		\item Metoda najmniejszych kwadratów - $mnk.m$
		\item Testowanie algorytmów regulacji - $regulacja.m$, przy czym używane przez niego funkcje poszczególnych algorytmów to: $funregnpl.m$, $funreggpc.m$ oraz $funregno.m$.
	\end{itemize}

	\textbf{Inne załączone pliki:}
	\begin{itemize}
		\item $daneucz.mat$ i $danewer.mat$ - wykorzystywane dane uczące i weryfikujące
		\item $g\_1.m$ i $g\_2.m$ - funkcje procesu
		\item $modelBFGS\_OE$, $modelBFGS\_ARX$, $modelNS\_OE$, $uczenieBFGS\_OE$, $uczenieBFGS\_ARX$, $uczenieNS\_OE$ - nauczone sieci neuronowe prezentowane w sekcji drugiej
		\item $siec.m$ - rekurencyjne liczenie wyjścia sieci neuronowej używane w ewaluowaniu modeli
		\item $snout.m$ - liczenie wyjścia sieci neuronowej dla zadanego wejścia, używane w funkcjach regulacji
	\end{itemize}